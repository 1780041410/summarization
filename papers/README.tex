\documentclass[UTF8]{ctexart}
\usepackage{minted}
\usepackage{xcolor}
\usemintedstyle[python3]{monokai}
\definecolor{bg}{rgb}{0.1.,0.14.,0.1}


\begin{document}
    This is a latex README File, we will use some addtional infromation to derive the model 
    here, notice that we will also give some insight to the model we see and the paper we have
    seen.
    
    This also include some code analysis for achiving the given algorithm and how will we use
    the pytorch to achive the models proposed in the papers.
    PyTorch 代码块如下所示

    % we may adjust the formatting of our coding here.
    \begin{minted}[linenos=true, mathescape, bgcolor =bg]{python3}
    import pandas as pd
    import os
    import sys
    import torch
    import torch.nn as nn
    import torch.nn.functional as F

    class MyModel(nn.Module):
        """[define some args]
        
        Parameters
        ----------
        nn : [pytorch Module]
            [description]
        
        """

        def __init__(self, in_dim, out_dim):
            def __init__(MyModel, self).__init__()
            hidden_dim_ = 5
            self.linear1 = nn.Linear(in_dim, hidden_dim_)
            self.linear2 = nn.Linear(hidden_dim_, out_dim)
        def forward(self):
            pass
        
    \end{minted}

    $$Y^2 = x^3+5\frac{x}{y^2}+4\times x + \Delta$$
\end{document}